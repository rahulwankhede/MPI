\documentclass[12pt,letterpaper]{article}
%\usepackage{fullpage}
\usepackage[top=2cm, bottom=4.5cm, left=2.5cm, right=2.5cm]{geometry}
\usepackage{amsmath,amsthm,amsfonts,amssymb,amscd}
%\usepackage{lastpage}
\usepackage{enumerate}
\usepackage{fancyhdr}
%\usepackage{mathrsfs}
\usepackage{xcolor}
\usepackage{graphicx}
\usepackage{listings}
\usepackage{hyperref}
\usepackage{float}

\hypersetup{%
  colorlinks=true,
  linkcolor=blue,
  linkbordercolor={0 0 1}
}
 
\renewcommand\lstlistingname{Code}
\renewcommand\lstlistlistingname{Code}
\def\lstlistingautorefname{Alg.}

\lstdefinestyle{C}{
    language        = C,
    frame           = lines, 
    basicstyle      = \footnotesize,
    keywordstyle    = \color{blue},
    stringstyle     = \color{green},
    commentstyle    = \color{red}\ttfamily
}

\setlength{\parindent}{0.0in}
\setlength{\parskip}{0.05in}

% Edit these as appropriate

\pagestyle{fancyplain}
\headheight 35pt
\lhead{Rahul Wankhede \\ MTech (Res), CDS}                 % <-- Comment this line out for problem sets (make sure you are person #1)
\chead{\textbf{\Large MPI Assignment}}
\rhead{DS 288 \\ Due: Nov 17th, 2019}
\lfoot{}
\cfoot{}
\rfoot{\small\thepage}
\headsep 1.5em

\begin{document}

\section{Methodology}

We make every process read its own part of the file into a local array.
Every process also reads the query file into a local query-array.
One element at a time is taken out from the query array and every process searches for it in their local array. If some process finds it, it sends a message, the global index at where the element was found, to every process (including itself). Note: the global index is calculated by adding the number of elements with processors before it to the local index. The process with rank 0 then writes the index to the output file (or -1 if search was unsuccessful). The next element is taken out from the query array and the procedure is repeated until all 50 elements have been searched.

\section{Execution Times}

\subsection{p = 1}

The average time taken for the execution of the program is 10.3 seconds

\subsection{p = 2}

The average time taken for the execution of the program is 5.6 seconds

Speedup = 1.84x

\subsubsection{p = 4}

The average time taken for the execution of the program is 3.0 seconds

Speedup = 3.43x

\subsubsection{p = 8}

The average time taken for the execution of the program is 4.1 seconds

Speedup = 2.57x

\section{Observations}


For different number of threads



\end{document}